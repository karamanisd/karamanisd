\chapter{Εισαγωγή}

Η Μηχανική Μάθηση (\en{Machine Learning}) αποτελεί τομέα της Τεχνητής Νοημοσύνης (\en{Artificial Intelligence}) και αναφέρεται στο σχεδιασμό και την υλοποίηση συστημάτων που μπορούν να εκπαιδευτούν και να παράγουν γνώση από ένα σύνολο δεδομένων. Για παράδειγμα θα μπορούσαμε από ένα σύνολο μηνυμάτων να εκπαιδεύσουμε ένα σύστημα ωστε να αναγνωρίζει ένα μήνυμα ως απειλητικό ή μη. Μετά το πέρας της διαδικασίας εκπαίδευσης, το σύστημα θα είναι σε θέση να αναγνωρίζει ένα οποιοδήποτε μήνυμα που δεν ανήκει στο σύνολο των ήδη γνωστών δεδομένων και να το χαρακτηρίζει ως προς το αν είναι απειλητικό για το χρήστη.

Αυτή η διαδικασία αποτελείται απο δύο βασικές έννοιες: την αναπαράσταση και τη γενίκευση. Ο όρος αναπαράσταση αναφέρεται στην προσαρμογή των δεδομένων (πχ το μήνυμα κειμένου) σε γλώσσα και μορφή κατανοητή για τον υπολογιστή. Αυτό επιτυγχάνεται με χρήση εξισώσεων και μεθόδων μετατροπής που συχνά μοιάζουν με τις λειτουργίες των νευρώνων. Ο όρος γενίκευση αναφέρεται στη δυνατότητα του εκπαιδευμένου συστήματος να λειτουργεί εξ ίσου καλά με δεδομένα πάνω στα οποία δεν έχει εκπαιδευτεί.

Η εξόρυξη δεδομένων (\en{data mining}) είναι βασικό κομμάτι της διαδικασίας εξαγωγής γνώσης από τα δεδομένα. Η άντληση πληροφορίας από ένα σύνολο δεδομένων συχνά άγνωστου περιεχομένου και η δημιουργία γνώσης απο αυτήν, αποτελεί μια πρόκληση που απαιτεί καλό σχεδιασμό και σε νοηματικό αλλά και σε προγραμματιστικό επίπεδο.

Προτού φτάσουμε σε αυτό το σημείο, καλή και απαραίτητη πρακτική αποτελεί το στάδιο της Προεπεξεργασίας (\en{Preproccessing}). Σε αυτό το στάδιο, εκτελούμε κάποιες διεργασίες στο σύνολο των δεδομένων ώστε να φτάσουν στην είσοδο του προγράμματος πιο "καθαρά", απομονώνοντας το θόρυβο με σκοπό να κρατήσουμε την πιο ουσιώδη πληροφορία. Στην ουσία, διατηρούνται μόνον οι πιο σημαντικές λέξεις στην πιο απλή μορφή τους και διαγράφονται άρθρα, συνηθισμένα ρήματα χωρίς σημαντική νοηματική συνεισφορά, σημεία στίξης και άλλες συχνά χρησιμοποιύμενες λέξεις, απαραίτητες για την επικοινωνία μεταξύ των ανθρώπων αλλά δίχως ιδιαίτερη επίδραση στη διαμόρφωση του νοήματος. Αυτό έχει ως αποτέλεσμα την εξοικονόμηση πόρων, υπολογιστικής ισχύος αλλά και την πιο στοχευμένη εκπαίδευση του συστήματος και φυσικά τη βελτίωση των αποτελεσμάτων.

Οι διεργασίες που επιτελούνται στο στάδιο της Προεπεξεργασίας είναι οι εξής:
\begin{itemize}
	\item Καθαρισμός Δεδομένων (\en{data cleansing})
    \item Ενοποίηση δεδομένων \en{(Data integration)}
    \item Μετασχηματισμός δεδομένων (\en{Data transformation)} και Διακριτοποίηση δεδομένων \en{(Data discretization)}
    \item Μείωση δεδομένων \en{(Data reduction)}
\end{itemize}


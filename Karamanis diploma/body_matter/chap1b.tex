\chapter{\en{Related work}}

Η χρήση τεχνολογιών \en{Machine Learning} και \en{NLP} φαίνεται να απασχολεί ιδιαίτερα την ιατροδιαγνωστική κοινότητα, αφού αποτελεί έναν τομέα που θα επωφεληθεί ιδιαίτερα από τα αποτελέσματα τους. Υποστηρίζεται από πολλούς ότι εάν η απόδοση της αγγίξει επιθυμητά αποτελέσματα τότε οι επιτυχείς διαγνώσεις θα μπορούν να επιτυγχάνονται σε σημαντικά μικρότερο χρονικό διάστημα ίσως ξεπερνώντας σε ακρίβεια τις απόψεις των γιατρών οι οποίοι πολλές φορές επηρεάζονται από τα συναισθήματα τους. Αρκετές επιστημονικές ομάδες έχουν ασχοληθεί με τον τομέα αυτό λαμβάνοντας αρκετά υποσχόμενα αποτελέσματα, όμως θα πρέπει να σημειωθεί ότι οι περισσότεροι από αυτούς συμφωνούν στο ότι οι τεχνικές αυτές δεν στοχεύουν στην αντικατάσταση των ιατρών αλλά θα μπορούν να προσφέρουν πολύ σημαντική πληροφορία διευκολύνοντας ιδιαίτερα την λήψη αποφάσεων.

\en{Οι \en{Po-Hao Chen, Hanna Zafar, Maya Galperin-Aizenberg \& Tessa Cook} στην εργασία τους \cite{related1_PaoHaoChen} ενσωματώνουν αλγόριθμους Επεξεργασίας Φυσικής Γλώσσας και Μηχανικής Μάθησης για την κατηγοριοποίηση της ογκολογικής απόκρισης στις αναφορές ακτινολογίας. Υποστηρίζουν ότι οι τεχνικές επεξεργασίας φυσικής γλώσσας (\en{NLP}) και μηχανικής μάθησης (\en{ML}) έχουν δείξει ότι εξάγουν με επιτυχία πληροφορίες από αναφορές μαγνητικής ή αξονικής τομογραφίας. Συνδυάζουν κάθε μία από τις τρεις τεχνικές \en{NLP} με πέντε αλγόριθμους \en{ML} για να προβλέψουν την ογκολογική διάγνωση του ασθενούς  χρησιμοποιώντας το μη δομημένο κείμενο της αναφοράς της και συγκρίνουν την απόδοση κάθε συνδυασμού. Οι \en{NLP} αλγόριθμοι που χρησιμοποιήθηκαν είναι οι \en{TF-IDF, TF}, και το \en{16 bit hashing}. Oι ML αλγόριθμοι από την άλλη είναι οι Logistic Regression, RDF, SVM, \en{BPM} και \en{NN}. Κατέληξαν στο ότι με όλες τις παραμέτρους βελτιστοποιημένες, το SVM είχε την καλύτερη απόδοση στο σύνολο δεδομένων δοκιμής, με μέση ακρίβεια 90,6 και βαθμολογία \en{F} 0,813. Η αλληλεπίδραση μεταξύ των αλγορίθμων \en{ML} και \en{NLP} και η επίδρασή τους στην ακρίβεια της ερμηνείας είναι πολύπλοκη. Η καλύτερη ακρίβεια επιτυγχάνεται όταν και οι δύο αλγόριθμοι βελτιστοποιούνται ταυτόχρονα.} 

Οι \en{Shiva Kazempour Dehkordi \& Hedieh Sajedi} επιχειρούν πρόβλεψη ασθένειας με βάση τη πρώτιστη συνταγογράφηση με χρήση μεθόδων εξόρυξης δεδομένων. \cite{related2_Shiva} Ο στόχος της έρευνάς τους είναι η χρήση μεθόδων εξόρυξης δεδομένων για την εξεύρεση γνώσης από ένα σύνολο δεδομένων που παρασχέθηκε από ένα ερευνητικό κέντρο. Αναλύοντας τα φάρμακα που αγοράστηκαν από τον κάθε ασθενή, η προτεινόμενη μέθοδος τους στοχεύει στο να προβλέψει τον τύπο του γιατρού στον οποίο έχει ανατεθεί ο κάθε ασθενής και το είδος της νόσου από την οποία πάσχει. Χρησιμοποιούν τρεις αλγορίθμους εξόρυξης δεδομένων, οι οποίες ήταν το δέντρο απόφασης, ο \en{Naïve Bayes} και ο \en{kNN}. Βλέποντας όμως ότι κανένας από αυτούς δεν λειτούργησε σωστά, εφαρμόστηκε ένας ταξινομητής στοίβαξης, ο οποίος αποδείχθηκε ότι έχει μεγαλύτερη ακρίβεια από τους προηγούμενους. Στην πρώτη έκδοση του συνόλου δεδομένων, τρεις διαφορετικοί βασικοί αλγόριθμοι που περιλαμβάνουν \en{kNN, Decision Tree και SVM} εφαρμόστηκαν για ταξινόμηση, ενώ στη δεύτερη έκδοση, τέσσερις διαφορετικοί, όπως \en{kNN, Decision Tree, Generalized Linear Model} και \en{Random Forest} χρησιμοποιήθηκαν στον \en{Stacking Operator}. Επιχειρούν έτσι αρκετά πειράματα για να συγκριθεί η απόδοση διαφορετικών τεχνικών εξόρυξης δεδομένων για την πρόβλεψη των ασθενειών και τα αποτελέσματα δείχνουν ότι το προτεινόμενο \en{Stacking Model} έχει υψηλότερη ακρίβεια σε σύγκριση με άλλες τεχνικές εξόρυξης δεδομένων όπως το \en{k-Narest Neighbor (kNN)}.

\en{Στον γειτονικό τομέα της κτηνιατρικής οι Abdullah Awaysheh, Jeffrey Wilcke, François Elvinger, Loren Rees, Weiguo Fan συμφωνούν ότι οι μέθοδοι μηχανικής μάθησης μπορούν να βοηθήσουν στις διαδικασίες λήψης ιατρικών αποφάσεων τόσο σε κλινικό όσο και σε διαγνωστικό επίπεδο. \cite{related3_Abdullah} Στην έρευνά τους ρίχνουν μια μηχανιστική ματιά σε τρεις αρχετυπικούς αλγόριθμους μάθησης — naive Bayes, δέντρα αποφάσεων και νευρωνικά δίκτυα — που χρησιμοποιούνται συνήθως για την τροφοδοσία αυτών των εργαλείων υποστήριξης ιατρικών αποφάσεων. Επίσης εστιάζουν τη παρατήρηση τους στα σύνολα δεδομένων που χρησιμοποιούνται για την εκπαίδευση αυτών των αλγορίθμων και εξετάζουν μεθόδους για επικύρωση, αναπαράσταση δεδομένων, μετασχηματισμό και σωστή επιλογή features. Απέδειξαν ότι η ποιότητα των δεδομένων εισόδου έχει μεγάλο αντίκτυπο στη διαδικασία μηχανικής μάθησης και στην απόδοση αυτών των συστημάτων και παρουσιάζουν αποδεικτικά στοιχεία ότι αυτές οι εφαρμογές βελτιώνουν την ακρίβεια των ιατρικών διαγνώσεων και συμβάλλουν σε καλύτερα αποτελέσματα ασθενών}

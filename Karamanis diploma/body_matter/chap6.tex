\chapter{Επίλογος}

\section{Αποτελέσματα}

Συγκεντρωτικά, τα αποτελέσματα όλων των μεθόδων φαίνονται στον παρακάτω πίνακα: 

\setlength{\arrayrulewidth}{0.5mm}
\setlength{\tabcolsep}{18pt}
\renewcommand{\arraystretch}{1.5}
\begin{table}[h]
    \centering
    \begin{tabular}{  | l | c | }
    \hline
    \multicolumn{2}{|c|}{Πίνακας Αποτελεσμάτων} \\
    \hline
    Αλγόριθμος & Ακρίβεια \\
    \hline
    \en{Logistic Regression} & 0.64 \\
    \en{Logistic Regression \& SMOTE} & 0.67 \\
    \en{Naive - Bayes} & 0.65\\
    \en{K Nearest Neighbors (KNN)} & 0.63\\
    \en{Support Vector Machines (SVM)} & 0.67\\
    \en{Word2Vec \& Neural Network}  & 0.68  \\
    \hline
    \end{tabular}
    \caption{Συγκεντρωτικός Πίνακας Αποτελεσμάτων Ακρίβειας}
\end{table}

\section{Συμπεράσματα}

Συμπερασματικά, το συγκεκριμένο \en{dataset} δεν είναι εύκολο να επεξεργαστεί με ακρίβεια καθώς δεν είναι «καθαρό».
Με διάφορες τεχνικές φτάνουμε σε καλύτερα αποτελέσματα, ωστόσο αυτό δεν ειναι εφικτό να εφαρμοστεί σωστά σε οποιαδήποτε περιγραφή περιστατικού.

Mέσα από τις διάφορες προσεγγίσεις, με διαφορετική αναπαράσταση δεδομένων και διαφορετικές τεχνικές, καταφέραμε να φτάσουμε σε κάποια ικανοποιητικά ποσοστά ακρίβειας. 

Το μεγαλύτερο ίσως εμπόδιο για την επίτευξη αυτού του στόχου ήταν η αντιμετώπιση της ανισορροπίας των δεδομένων, δηλαδή η άνιση κατανομή τους στις διάφορες κλάσεις, η επικάλυψη των κλάσεων, η σωστή προεπεξεργασία των κειμένων ώστε να απομονωθεί η χρήσιμη πληροφορία και η σωστή παραμετροποίηση του νευρωνικού δικτύου για την αποτελεσματική εκπαίδευση του μοντέλου. 

\section{Μελλοντικές Επεκτάσεις}
Ωστόσο, τα ποσοστά ακρίβειας που επιτεύχθησαν, αποτελούν αισιόδοξο σημάδι για μελλοντική δουλειά. Δημιουργώντας στο μέλλον ένα πιο ισορροπημένο και ακριβές \en{dataset}, θα μπορέσουμε να εκπαιδεύσουμε καλύτερα το σύστημα μας ωστε να αναπτύξουμε ένα ακόμα πιο ισχυρό και αποτελεσματικό εργαλείο.
Πέρα από την πρόβλεψη ιατρικής ειδικότητας, μέσα απο την περιγραφή ιατρικού περιστατικού θα μπορούσε να δημιουργηθεί πρόγνωση για το μέτρο του επείγοντος, τη φαρμακευτική περίθαλψη που ενδεχομένως χρειστεί ή την πιθανότητα να χρειαστεί εισαγωγή και νοσηλεία, αντικείμενα και αποφάσεις που απασχολούν ιατρικό προσωπικό χωρίς απαραίτητα να χρήζουν ιατρικών γνώσεων. 

Εν κατακλείδι, η δημιουργία χρήσιμων, ισχυρών και αποτελεσματικών εργαλείων στα χέρια ιατρών, ερευνητών και επιστημόνων αποτελεί ένα πεδίο με μεγάλο φάσμα προκλήσεων και προς επίλυση προβλημάτων, στοχεύοντας πάντα στην υποστήριξη και όχι στην αντικατάσταση επιστημονικά καταρτισμένου προσωπικού. 


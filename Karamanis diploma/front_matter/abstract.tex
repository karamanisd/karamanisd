\begin{abstract}
Η εκπαίδευση ενός υπολογιστικού συστήματος στην νοηματική αναγνώριση κειμένου και την εξαγωγή συμπερασμάτων από αυτό, αποτελεί μια πρόκληση για την επιστημονική κοινότητα και την έρευνα.

Στόχος του συγκεκριμένου ερευνητικού πεδίου που δημιουργήθηκε από αυτήν την πρόκληση είναι η δημιουργία κατάλληλων εργαλείων για την υποστήριξη, υποβοήθηση και διευκόλυνση του ανθρώπινου δυναμικού.

Στην παρούσα διπλωματική εργασία δίνεται έμφαση στην χρήση ιατρικού κειμένου ως πηγή δεδομένων, στη σωστή προεπεξεργασία και στην κατηγοριοποίηση του σε κάποια ιατρική ειδικότητα με σχεδιασμό και χρήση του κατάλληλου μοντέλου. Οι μέθοδοι που χρησιμοποιούνται για την αναπαράσταση των λέξεων με διανύσματα είναι η μέθοδος \tl{tf-Idf} και ο αλγόριθμος ενσωμάτωσης λέξεων \tl{Word2Vec}. Οι αλγόριθμοι ταξινόμησης που χρησιμοποιήθηκαν για την εξόρυξη γνώσης στην υλοποίηση είναι η Λογιστική Παλινδρόμηση \tl{(Logistic Regression)}, ο αλγόριθμος \tl{Naïve-Bayes}, ο αλγόριθμος \tl{Supported Vector Machine (SVM)}, ο αλγόριθμος \tl{k Nearest Neighbors (kNN)} και τα νευρωνικά δίκτυα βαθιάς μάθησης \tl{(Neural Networks)} με χρήση της βιβλιοθήκης \tl{Tensorflow/Keras}.

Η εξαγωγή συμπερασμάτων από το υπολογιστικό σύστημα, με είσοδο ως δεδομένο μόνον την περιγραφή ενός ιατρικού περιστατικού, ίσως γίνει η απαρχή ώστε στο μέλλον να αποτελέσει σημαντικό εργαλείο στα χέρια των ιατρών, των ερευνητών και των επιστημόνων.

   \begin{keywords}
   Εξόρυξη κειμένου, κατηγοριοποίηση κειμένου, νευρωνικά δίκτυα, λογιστική παλινδρόμηση  \tl{text mining}, \tl{multiclass classification}, \tl{clinical text}, \tl{tf-idf}, \tl{word2vec}, \tl{deep learning}, \tl{logistic regression}
   \end{keywords}
\end{abstract}



\begin{abstracteng}

\tl{Training a system to understand natural language and extract knowledge out of natural text is a challenge for the scientific community. The purpose of this research field is to create tools to support and simplify the work of human resources.}

\tl{This thesis focuses on medical text as a source, aiming to correctly preprocess it and train machine learning models to automatically identify the medical specialty to which it should be classified.}

\tl{The techniques used to vectorize words are the inverse terms frequency (Tf-Idf) and the Word2Vec algorithm. Classification algorithms used are Logistic Regression, Naïve-Bayes, k Nearest Neighbors (kNN), Supported Vector Machines (SVM) and Neural Networks, using the Tensorflow/Keras library.}

\tl{Creating a system able to extract knowledge out of medical text using only a medical transcription written in natural language may be a strong basis for future development of powerful tools for doctors, scientists and researchers.}

   \begin{keywordseng}
    \tl{Machine Learning, Artificial Intelligence, tf-idf, Word2Vec, Logistic Regression, Naïve-Bayes, kNN, SVM, Neural Networks, text mining, text classification, medical text, NLP}
   \end{keywordseng}

\end{abstracteng}